\documentclass[final,2p,times]{elsarticle}

\usepackage{hyperref}
\usepackage[utf8]{inputenc}

\journal{Intelligent Environmental Data Analysis and Pollution Management}

\begin{document}

\begin{frontmatter}

\title{Aire Guru: Making Pollution Data Accessible, Relevant, and Compelling.}

\author{Luz Morales}
\address{Universidad Internacional de La Rioja}
\ead{mariadelaluz.morales489@comunidadunir.net}

\begin{abstract}
Goveramental open data portals make a huge effort to collect data and offer it free to their citiziens.
However, this data is often wasted, since it is not always in an accesible, relevant and compelling format
for the averange citizien.
We created a complete system, which collects the data periodically, store it to be able to consult the historical records,
process it to extract the relevant information, personalize it to engadge the users with its importance in each individual
case, visualize it in a format everybody can understand and provide it online to facilitate its accesibility. Furthermore,
descriptions all the data and how it has been processed are also provided to raise the awareness about the value
of the usage of this data. This webtool is Aire Guru and has been sucesfully tested, using the Air Quality data provided
by the Málaga open data portal\footnote{\url{https://datosabiertos.malaga.eu/}} in the city of Málaga.


Nowadays we are surrounded by many sources of pollution which contaminates the air, the air 
which we breath. This contamination is the cause of many and different symptoms, from a simple eye irritation 
to an extreme case, the death. Specially for the sensitive groups like children and elders.
The WHO (World Health Organization) estimates a 4.2 million deaths
\footnote{\url{https://www.who.int/airpollution/ambient/en/}} per year due the air pollution.
The list is very large, the air pollution affect to people with asthma, cardiopathies, allergies,
long cancer, EPOC, leukemia, pneumonia, autism, neurologic pathologies,etc.
It does not only aggravate the symptoms but it even can be the cause of it, like the case of brain
damage\footnote{\url{https://www.cronicabalear.es/2018/07/la-contaminacion-ambiental-causa-
enfermedades-neurologicas-y-envejecimiento-del-cerebro/}} 
in the newborns caused by the air pollution consumed by the mother.

To be able to control the level of exposure to the air pollution, we need to know what are these levels
in the different areas where we frecuently are and the variation during the different slots.
To make this information as accessible as possible, it should be open to everybody and the format the levels
are shown, should be simple and complete to be understandible for the averange people.

There are already many tools available, however, they do not succeed to answer these questions.
The most common problems are:

\begin{itemize}
\item Obsolete measurements. The measures need to be taken periodically, since it could be a huge difference
between different moments of the day. Otherways, it will not be relevant.
\item Too few measurements. A city which offer only few measure is not compelling for the users.
\item Considering a measurement for a huge area. A measurement should represent the measure of an 
specific area, one single measurement for an entired city is not a valid or relevant measurement.
\item Bad representation of the information. The information is represented on a table, making really dificult
to locate the measurement in a map, others is represented in a really specific format, no understandible for the averange user.
This result in a no compelling visualization.
\item Insufficient information. Tools which show one single value as a number or a color. This information is not
enough for the user to have the control to the situation. This visualizations are not really compelling. 
\end{itemize}

Our webtool, Aire Guru \footnote{\url{https://airquality.guru/}}, solves the current deficiencies. 

Collects and process the data. This data is published hourly and it comes from the measurement stations placed along the 
city of Málaga, covering its entired extension divided by areas of $100m^2$.
Even if the data is completed and updated, a collection of measurements does not mean anything if it is not processed, 
only after this step, it will be possible to extract the relevant information. In the context of air pollution, the information 
is represented by the AQI (Air Quality Index). 

Aire Guru represents the information in a simple and clear way. Making possible to spot the most dangerous places 
and allowing us to make comparisons. Furthermore, it has personalized information, like the most relevant pollutants for our
health in the case of a medical condition and the pollution we are exposed to on real time and the past. 
This last feature, monitor and control the pollution we are exposed on an individual level, is possible if we record periodically the
pollution we are exposed to. More personalized, more compelling.

Thanks to its periodicity and granularity, it facilitates us the calculation of the exposure of the different pollutants in an 
individual level. This implementation avoids costs for the final user which does not need to carry a measurement station with them. 
Furthermore, it adds more value to the data.

Aire Guru is available online free of charged, on a legible format, bringint it closer to the final user and provides the all 
information to help the final user to understand the air pollution, sources, consecuencies and how is calculated. Hence, the data 
is fully accesible to the users.

It has been sucessfully tested with 14 subject in the Spanish city of Málaga. The survey shows a high satisfaction on its usage.
The information is clear and complete.More than the half of the subjects did not know how the air quality was measure and how it 
could affect to their health, they learned it using the webtool and four of the subjects discovered that they have a medical 
condition which can be affected by the air pollution. Aire Guru raised the awareness on the test subjects.
    
\end{abstract}

\begin{keyword}
Air Quality Index \sep Pollution \sep Monitoring \sep Visualization \sep Analysis 
\end{keyword}

\end{frontmatter}

\end{document}