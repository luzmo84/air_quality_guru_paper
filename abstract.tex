\documentclass[final,2p,times]{elsarticle}

\usepackage{hyperref}
\usepackage[utf8]{inputenc}

\journal{Intelligent Environmental Data Analysis and Pollution Management}

\begin{document}

\begin{frontmatter}

\title{Aire Guru: Making Pollution Data Accessible, Relevent, and Compelling.}

\author{Luz Morales}
\address{Universidad Internacional de La Rioja}
\ead{mariadelaluz.morales489@comunidadunir.net}

\begin{abstract}

Nowadays we are surrounded by many sources of pollution which contaminates the air, the air 
which we breath. Some sources are known, like the pollutants produced by
the industries or the cars but others like the CO producers by the heaters at home or 
particulate material by the aerosol sprays at home are unknown for most of the people.

This contamination is the cause of many and different symptoms, from a simple eye irritation 
to an extreme case, the death.
The WHO (World Health Organization) estimates a 4.2 million deaths
\footnote{\url{https://www.who.int/airpollution/ambient/en/}} per year due the air pollution.
The list is very large, the air pollution affect to people with asthma, cardiopathies, allergies,
long cancer, EPOC, leukemia, pneumonia, autism, neurologic pathologies.
It does not only aggravate the symptoms but it even can be the cause of it, like the case of brain
damage\footnote{\url{https://www.cronicabalear.es/2018/07/la-contaminacion-ambiental-causa-
enfermedades-neurologicas-y-envejecimiento-del-cerebro/}} 
in the newborns caused by the air pollution consumed by the mother.

Furthermore, is extremly dangerous for the sensible groups, like children, elders and people which 
different pathologies.
\\~\\
Do we know the level of air pollution we are exposed to? Is it anyway we could measure it? Could 
we avoid highly contaminated areas?
Can I have an accurate idea about the air quality around me without being an expert?
\\~\\
A tool which present us the AQI (Air Quality Index) around us would be a possible solution. 
This tool should represent the information in a simple and clear way, 
shows us the pollutants which affects us the most in our health for each particular case, spot the 
most dangerous places and allows us to make comparisons. 
\\~\\
There are already many tools available, however, they do not succeed to answer these questions.
The most common problems are:
\begin{itemize}
\item Obsolete measurements. To provide an accurate AQI, the measures need to be taken periodically, 
since it could be a huge difference between different moments of the day.
\item Too few measurements.
\item Considering a measurement for a huge area. A measurement should represent the measure of an 
specific area, one single measurement for an entired city is not a valid measurement.
\item Bad representation of the information. The information is represented on a table, making really dificult
to locate the measurement in a map, others is represented in a really specific format, no understandible for the averange user.
\item Insufficient information. Tools which show one single value. This is usually a number or color with the general AQI. 
\end{itemize}

One of the ways to monitor and control the pollution we are exposed on an individual level, is to record periodically the pollution we are exposed to. 
\\~\\
Our proposal is a webtool available online for everybody which display the general AQI in the exactly 
point where the subject is on real time, with an accuracy of $100m^2$ and the value of each pollutant which
is present in that point. 
This information can be filtered by medical condition, showing the most relevant pollutants.
Furthermore, it should be possible to see our personal records, with all the pollutants we have 
been exposed until now.
It will be ideal that the user do not need any extra device to take the measurements. Avoiding costs for the users and
the need of carrying a measurement station with them. It also should be possible to visualize the pollutions in the past, at least for the last year and 
changes on the pollutants per areas. The tool should be completed with definitions about pollutants, sources, medical conditions and 
what means the AQI and how it is calculate.
\\~\\
This webtool, Aire Guru \footnote{\url{https://airquality.guru/}}, is created and it has been 
tested with 14 subject in the Spanish city of Málaga.The measurements are taken from the official open data portal of Málaga
\footnote{\url{https://datosabiertos.malaga.eu/}}. The most relevant result is the high satisfaction of the testers, the information is clear and complete.
More than the half of the subjects didn’t know how the air quality was measure and how it could 
affect to their health,
they learned it using the webtool and four of the subjects discovered that they have a medical 
condition which can be affected by the air pollution.
    
\end{abstract}

\begin{keyword}
Air Quality Index \sep Pollution \sep Monitoring \sep Visualization \sep Analysis 
\end{keyword}

\end{frontmatter}

\end{document}