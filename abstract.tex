\documentclass[final,2p,times]{elsarticle}

\usepackage{hyperref}
\usepackage[utf8]{inputenc}

\journal{Intelligent Environmental Data Analysis and Pollution Management}

\begin{document}

\begin{frontmatter}

\title{Aire Guru: Making Pollution Data Accessible, Relevant, and Compelling.}

\author{Luz Morales}
\address{International University of La Rioja}
\ead{mariadelaluz.morales489@comunidadunir.net}

\begin{abstract}
Government open data portals make a huge effort to collect data and make it available to their citizens.
However, this data is often wasted, since it is not always in effectively accessible, relevant and compelling
for the average citizen. Air quality data is one such example.

Urban residents are surrounded by many sources of air pollution, and it is the direct cause of many different symptoms, ranging from simple eye irritation 
to, in extreme cases, death. Particularly for high-risk groups like children and the elderly.
The WHO (World Health Organization) estimates 4.2 million deaths
\footnote{\url{https://www.who.int/airpollution/ambient/en/}} per year due to air pollution.
Air pollution has a significant affect on people with asthma, cardiopathies, allergies, and neurological pathologies.

Air pollution not only aggravates existing diseases, but can also be an initial cause of them, such as in the case of foetal brain
damage\footnote{\url{https://www.cronicabalear.es/2018/07/la-contaminacion-ambiental-causa-
enfermedades-neurologicas-y-envejecimiento-del-cerebro/}} 
caused by the mother's exposure to air pollution.

To be able to control the level of exposure we need to know the levels
in the specific locations that we frequent, and the variation during specific times.
To make this information as accessible as possible, it should be freely and publicly available, and the presentation
must be simple enough to be understandable by the average citizen.

We have created a tool - Aire Guru - which collects air pollution data periodically, stores it for historical use, 
processes it to extract relevant information, personalizes it to engage the user with regard to their individual
circumstance, visualizes it in an easy to understand format, and provides it online to facilitate its accessibility. Furthermore,
descriptions of all the data and how it has been processed are also provided to raise public awareness of the value
of this data. Aire Guru has been successfully tested, using the Air Quality data provided
by the Málaga open data portal\footnote{\url{https://datosabiertos.malaga.eu/}} in the Spanish city of Málaga.

There are already many tools available, however they have various failings.

The most common problems are:

\begin{itemize}

\item Obsolete measurements. Measurements need to be taken regularly, since there can be huge differences
between pollution levels at different times of the day.

\item Limited geographic coverage. The data must cover a reasonable proportion of areas that people spend significant time in.

\item Insufficiently granular measurements. Measurements must be at a reasonably fine level of granularity. One single measurement for an entire city is not useful.

\item Poor presentation. Often the information is presented in an uninterpreted form, making it difficult for users to visualize, especially in a geographic sense.

\item Poor discrimination and interpretation. Many tools show individual values as a number or a colour. This information is not
enough for the user to take control of their exposure. Such visualizations are not really compelling. 

\end{itemize}

Our webtool \footnote{\url{https://airquality.guru/}}, solves the deficiencies described above.

Measurements are published hourly and come from measurement stations placed throughout the 
city of Málaga, covering entire urban region at a granularity of $100m^2$

The information is presented in a clear and simple manner, making it possible to see the most polluted regions,
and allowing the user to make comparisons both geographically and historically. It provides personalised information, such as the most relevant pollutants for a user's particular medical conditions, and can track
the pollution they are exposed to both in real time and over a historical period, by linking with location data from their mobile device. 

The system has been successfully tested with 14 subject in the Spanish city of Málaga. The survey showed a high degree of satisfaction, particularly regarding the clarity and completeness of the the information and analysis.
More than the half of the subjects were not aware of how air quality is measured and how it 
can affect their health. Four of the subjects discovered that they have a medical 
condition which can be affected by air pollution.
    
Aire Guru is available online, free of charge.
\end{abstract}


\begin{keyword}
Air Quality Index \sep Pollution \sep Monitoring \sep Visualization \sep Analysis \sep BigData
\end{keyword}

\end{frontmatter}

\end{document}
  
